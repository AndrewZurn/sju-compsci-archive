%% CS330 Report template
%% Created: Sept 17, 2012

\documentclass{article} %% {article} or {report}
\bibliographystyle{plain}
%% \usepackage{cite}


\addtolength{\textheight}{1in}
\addtolength{\voffset}{-.33in}


\title{Assignment 2: Report}
\author{Andrew Thom\\ Andrew Zurn\\ Chris Norby\\ Isaiah Schultz}
\date{September 21, 2012}

\begin{document}
\maketitle
\newpage

\tableofcontents
\newpage

%% More info on sectioning: 
%% http://en.wikibooks.org/wiki/LaTeX/Document_Structure#Sectioning_Commands
 
\section{Introduction \& Purpose}
\label{sec:intro}
%% No more than two paragraphs. Probably can throw in Problem, Background, 
%% Purpose, Scope, Objectives, and Review of Literature in this. Make new 
%% sections as if/as needed.
The purpose of this project was to help the four members of this group become
comfortable with writing technical reports for \textit{CSCI 330: Software 
Engineering}, an upper-devision computer science course offered at \textit{Saint
John's University} in Minnesota.\par

To demonstrate this, the report that follows represents a hypothetical situation
involving our team of experts giving the now quite popular \textit{Netflix} 
recommendations for building the software architecture behind their product. As
experts, we will be referencing architecture prerequisites from the course
textbook ``Code Complete.'' (See Section \ref{sec:materials} for 
details).\par

Please note that while some of this report was compiled using 
information attained from good, old-fashioned research, the majority of it is
based on the collective imaginative ingenuity of the team.

\section{Materials}
\label{sec:materials}
%% List Materials, equpment, and facilities used. Use detail (brand names, model
%% numbers, specs, etc).
The following is a brief explanation of the resources used for this project.\par
Our main reference point from which we began was the course textbook ``Code
Complete'' (Second Edition) by Steve McConnell. 10 fo the 20 architecture
prerequisites were chosen from section 3.5 of the text for discussion in this 
report. The full list of prerequisites can be found on pages 45-53. Our personal
in-class notes were also used to supplement this material.\par

For our research and composition of this report, each of our personal computers
were used. They all had varying specifications, but the specific specs of the
computers did not in any way affect the outcome of this project. Because of
this, further details regarding them will be omitted from this report.\par

As a further learning tool, we chose to use \LaTeX to compose this report.
Primarily, the Eclipse plugin version of \LaTeX was used.\par

Supplied by Dr. Mike Heroux, instructor of the course, we used an online 
description as a guide for formatting this report\cite{howto}.


\section{Methods}
%% Any theory and literal steps used during project.
Because the goal of this project was to become familiar with writing technical
reports, we started by reading through the guidelines that provided. We 
discussed ways that we could combine any of the points outlined on the website
so to ensure that each section had adequately substantial content.\par

For example, ``Introduction,'' ``Problem, background,'' and ``Purpose,
objectives, scope'' were decidedly combined into our Section~\ref{sec:intro}. As
individual sections, we felt that they would contain a mere sentence or two 
which felt like a lot of wasted space. Therefore those sections were logically
combined.\par

Our next step was to review the 20 architecture prerequisites mentioned in 
Section~\ref{sec:materials} of this report. As assigned by Dr. Heroux, we picked
10 of the most relevant prerequisites to discuss within the scope of our
hypothetical situation:
\begin{enumerate}
  \item Program Organization
  \item Major Classes
  \item Data Design
  \item Performance
  \item Business Rules
  \item UI Design
  \item Security
  \item Scalability
  \item Internationalization
  \item Input/Ouput
\end{enumerate}
With similar reasons for combining a number of suggested sections into one 
introductory section for this report, we decided to combine ``Data Design'' and
``Performance'' into one section as well. As can be seen in 
section~\ref{sec:data_perf}, this decision made was logical for this situation.
\par

We finally discussed how each prerequisite applied to Netflix's software 
product. Section~\ref{sec:results} outlines the results of that discussion.

\section{Results}
\label{sec:results}
%% Summaries of data, pics/graphs/charts and shit.
\subsection{Program Organization}
Netflix can first be broken down into two major systems, one system that covers 
the online instant streaming, and the other one that covers the DVD-ship-to-home 
rental service.  Both are comprised of a various major subsystems that are bulk 
of the current end-user service that is known to us as commercial users.\par

First we can break down the online streaming service.  This service is comprised 
four subsystems. The first component of the online service is video storage, or 
how the videos that can be viewed by the Netflix user are stored.  Our next 
component is the user�s profile management, which takes care of storing valuable 
data that expands from financial information to what is currently in the user�s 
queue.  From there, there is another subsystem that is involved in managing the 
connection between the user interface and the database.  Finally, there is the 
UI and service management, which controls the way content is viewed, whether 
that be through the website or through an application stored on the local 
system.\par

The DVD rental service is not nearly as involved as the online service is, as it 
is comprised of two major systems.  One system, which manages the inventory 
currently in the warehouses, and the request and shipping service, which 
fulfills user�s requests for a DVD, and then handles the shipping of that DVD.
\par


\subsection{Major Classes}
We can further break down the subsystems described above, mainly the online 
streaming service, into major object classes that take care a lot of the 
individual tasks and processes involved in the Netflix online system.  There are 
three main classes within the Netflix system, those being of class Video, User, 
and DB Management.  From those classes, the Video class can be broken down into 
TV Show and Movie subclasses.  User is a super class for subclasses Customer and 
Admin, and Database Manager is a class that does a lot of backend work for 
maintaining and serving different content.\par 
%%MAYBE A GRAPH SHOWING THE CONNECTIONS?

\subsection{Data Design \& Performance}
\label{sec:data_perf}
As Netflix has a high traffic rate, and because it provides services in large file streaming, 
the data design and performance of the software is state-of-the-art, in order to maintain customer 
satisfaction and the quality of their service.  As such, the service response time is very minimal, 
because the company various sites across the world to retrieve and send information to its customers.  
Because Netflix insures the backend of its services is well supported, the only bottle neck in 
performance that can be found is the user�s Internet speed.  Netflix tries to condition its media 
using the latest video formats and codecs, in order to help ease the problematic bottleneck that 
it can do little in addressing.\par

\subsection{Business Rules}
Many of the business rules specified in this section will have to come not from 
us, the technical experts, but from the business department of Netflix.  We can, 
however, take a guess at some of the things they might say.  

There must be a minimum acceptable quality (bitrate) for all videos stored in 
the database.  The verbiage and terminology across the system must be 
consistent.  For example, if we refer to a 'movie' on one page of the website, 
the same object must not be referred to as a 'video' elsewhere.  This will avoid 
confusion by users as well as programmers.  The styling and color scheme across 
the website as well as any mobile apps must be consistent.  This will help the 
branding of the company.  Also, a consistent header and footer must appear on 
every page.

Each of these things must be specified in business rules files, as opposed to 
being written directly into the code.  This will make it easier to change any of 
these values.

\subsection{UI Design}
\label{subsec:UI}
As mentioned in the business rules section, the User Interface must be 
consistent across every page with regard to colors, styling, fonts, etc.  
Additionally, the files that specify the UI must not contain any logic.  Doing 
so would violate encapsulation and make any changes to either the UI or the 
logic very difficult.

There will be aspects of the UI that will be used many times across different 
pages of the website, such as headers, footers, menu bars, and frames for movie 
embedding.  It is vital that these common parts are only written once, and then 
linked to each time thay are used.  If the header, for example, is copy-pasted 
to the top of each UI file and a change of the header is required, the change 
will have to happen in many different places, which will lead to mistakes and 
inconsistencies.

\subsection{Security}
The videos for Netflix will be using Microsoft Silverlight.  This 
will aid us in keeping the copyrighted content on the website secure, due to 
some built-in security functionalities.  For example, users will not be able to 
directly download the movies, but only stream them to the website.  

Usernames and passwords must be standardized, with a given number of characters 
as well as possibly numbers, capital letters, and special characters.  Also, the 
master database that holds the movies must be fully encrypted, with all standard 
security features implemented.  All queries to the database must be done with 
prepared statements so as to avoid SQL injection by Dr. Rahal.

\subsection{Scalability}
As Netflix has grown from a novelty service for those who wanted the convenience 
of having DVDs delivered right to their mailbox, the company has had to adapt 
their business model to meet increasing demand.  Also, with the wide adoption of 
smartphones such as iPhone and Android devices, and with video game consoles 
becoming more like a home media center device, Netflix has had to create ways to 
not only deliver DVDs to a mailbox, but also to instantly deliver them to ones 
living room or a device in their pocket.\par

In doing that, Netflix has created mobile applications that run on iOS (both 
iPhone and iPad) and Android, and has also worked with Microsoft, Sony, and 
Nintendo to integrate the Netflix service into their respective video game 
consoles.  This has allowed Netflix users to have access to their Netflix 
account anywhere at any time.\par

Also, Netflix has used a recommendation engine in order to help its users find 
more movies that they may like based upon the films they have watched, and the 
ratings that they have given to other movies.  While the engine was producing 
good results, Netflix wanted it to become more accurate.  They wanted teams of 
programmers to take their sample data, and use data mining techniques to come 
up with better algorithms that would predict movies better than its system, 
called \textit{Cinematch	}.  The contest, called the \textit{Netflix Prize}, 
ended up with one team that had successfully improved over the 
\textit{Cinematch} algorithm by more than 10\%.  That team, called BellKor's 
Pragmatic Chaos, received a prize of \$1,000,000.  

\subsection{Internationalization}
As Netflix's Instant Streaming service gained popularity in the United States, 
many other people around the world expressed interest in the service.  
Netflix's first international venture began on September 22, 2010, when they 
launched instant streaming in Canada.  By September 2011, Netflix instant 
streaming was introduced into the Caribbean, Mexican and South American markets.  
In January 2012, the service was also launched in the United Kingdom and 
Ireland, marking the first time that the service has gone overseas.  Netflix has 
also announced that they will be launching the service in the Scandinavian 
countries of Norway, Denmark, Sweden, and Finland before the end of 2012, but as 
of September 21, 2012, that service has yet to launch.  Currently, all of 
Netflix's operations outside of the United States include \emph{only} the 
instant streaming service, not the DVD-by-mail service.\par

Because of the contracts that Netflix has negotiated with the content 
distributors, the service is not accessible outside of the countries where it 
has been officially launched.  Netflix uses IP address geolocation to determine 
from where a user is accessing the service.  If the user is in a location not 
supported by Netflix, the instant streaming service is not available.

\subsection{Input/Output}
There are many different inputs and outputs to the Netflix service.  The movies 
must be added to the Netflix database, which is done manually.  Also, the users 
must use the Netflix web interface to sign up for the service, enter their 
shipping and billing information, and also determine what movies they want to 
add to their DVD and Instant Queues.  The information on how this is done is 
handled in subsection \ref{subsec:UI}.\par

The most important output of the Netflix service is, obviously, the movies that 
are instant streamed to its users' computers, smartphones, video game consoles, 
or other devices that support Netflix playback.  When playing back on the 
computer, Netflix uses the Microsoft Silverlight technology to stream the video 
while encoding it to remain secure.  Also, the web interface displays 
information about the videos, including a description of the film, the cast, and 
its current rating.  Administrative tasks are also output via the web interface.  
These tasks include handling the user's profile, updating personal information, 
and adding or changing payment details.

\section{Conclusions}
%% Usual interpretation of what the pics/graphs/charts and shit means. Also
%% possibly include recommendations for future?
This concludes our report. While we are sad to relinquish our roles as experts
for Netflix, we take solace in knowing that we have completed our first 
technical report as a team. While the content may not be as perfect as content
provided by full-time experts for Netflix, we have a very good grasp on how to 
apply the structure of this report to real, practical applications for the
\textit{Software Engineering} course this semester. As a team, we would like to
say, ``Assignment 2: Mission accomplished!''

\newpage
\addcontentsline{toc}{section}{References}
\bibliography{assignment2}

\end{document}